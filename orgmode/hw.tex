% Created 2020-06-15 Mon 02:03
% Intended LaTeX compiler: xelatex
\documentclass{homework}
\usepackage{graphicx}
\usepackage{grffile}
\usepackage{longtable}
\usepackage{wrapfig}
\usepackage{rotating}
\usepackage[normalem]{ulem}
\usepackage{amsmath}
\usepackage{textcomp}
\usepackage{amssymb}
\usepackage{capt-of}
\usepackage{hyperref}
\class{CS 3141: Prof. Kamil's Algorithm Analysis}
\address{Bayt El-Hikmah}
\lstset{language=python}
\author{Musa Al`Khwarizmi}
\date{\today}
\title{Homework in Org-mode}
\hypersetup{
 pdfauthor={Musa Al`Khwarizmi},
 pdftitle={Homework in Org-mode},
 pdfkeywords={},
 pdfsubject={},
 pdfcreator={Emacs 26.3 (Org mode 9.1.9)}, 
 pdflang={English}}
\begin{document}

\maketitle
This is a \underline{demonstration} of my homework \LaTeX{} class. It is an extension of the \texttt{amsart} and should have all of its functionality. These are some of the set symbols: \(\bC \supset \bR \supset \bQ \supset \bZ \supset \bN\), then some Greek and other mathematical symbols are, \(\al, \ep, \p, \ra, \Ra, \injective, \surjective, \bijective\).


\begin{question}[99]
Prove that \(\exists (x,y) \in \bZ\) such that \(x+y = 4\).

\begin{proof}
Four is the sum of two integers.

\(1,3 \in \bZ\) and \(1+3=4\).
\end{proof}
\end{question}
\end{document}